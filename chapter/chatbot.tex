\section{Chatbot}

\subsection{Chatbot}
Okeee.., kali ini kita akan mulai memperkenalkan apa itu chatbot dan potensi yang akan didapatkan jika kita membangun sebuah chatbot. Chatbot merupakan kata gabungan dari kata dasar berbahasa inggris yaitu \textit{chat} dan \textit{bot}, arti kata chat yaitu perbincangan atau komunikai antara 2 orang atau lebih, chat terbagi menjadi 2 yaitu group chat atau perbincangan yang dilakukan secara berkelompok, atau personal chat atau chat yang dilakukan hanya 2 orang saja dan tidak lebih. Bot merupakan potongan kata dari kata robot yang artinya adalah sebuah system yang dibangun untuk membantu manusia dalam sebuah pekerjaan. Lalu arti chatbot adalah sebuah robot/system yang dapat berinteraksi dengan manusia melewati perbincangan/komunikasi 2 arah bisa dalam group chat maupun personal chat. Sehingga chat yang dilakukan adalah orang dengan robot tetapi seakan-akan yang membalas pesan tersebut seperti manusia, padahal yang membalas itu adalah robot yang sudah dirancang sedemikian rupa dengan aturan yang sudah ditentukan oleh \textit{programmer} atau orang yang membangun otak dari robot chat itu sendiri.

\subsection{NLP dan Rule Based}

\subsubsection{NLP}
Natural Language Processing (NLP) adalah sebuah cabang keilmuan yang ada pada Artificial Intelligence (AI). NLP adalah sebuah keilmuan yang digunakan untuk interaksi antara manusia dengan mesin atau robot yang melewati perantara lingustik atau kata-kata. Beberapa hal yang akan dilakukan oleh NLP adalah membaca kalimat, mengekstrak kalimat menjadi kata-kata, mengartikan kata demi kata, memahaminya apa yang dimaksud, dan akan menghasilkan sebuah respon.

\subsubsection{Rule Based}
Rule Based adalah sebuah kata yang telah ditentukan oleh programmer seperti apa aturan yang harus ditetapkan agar chatbot akan merespon dengan aturan yang sudah ditetapkan, contoh: ketika programmer menetapkan ketika user chat "halo" maka chatbot hatus merespon "halo, juga", ketika user mengetikkan "hai" maka chatbot tidak akan merespon hal tersebut karena yang sudah di program adalah kata "halo" bukan kata "hai", maka berbeda dengan NLP yang mana setiap kata yang dikirim oleh user akan dipahaminya lalu akan diberikan sebuah respon oleh chatbot.